\section{Simplifying the expression \texorpdfstring{$1-w_p(\phi)^2$ from $y_p(\phi)$}{}}

\textit{Objective:} obtain the simplest expression for $1-w_p(\phi)^2$ in terms of $\phi,\beta$.

The original expression for $1-w_p(\phi)^2$ is below, with the denominator removed for simplicity.

{\footnotesize\begin{align}
    (3\pi+8)^2\beta^4\sin^2(2k)(1-w_p(\phi)^2)&=\underbrace{{(3\pi+8)^2\beta^4\sin^2(2k)-\left[(3\pi+8)^2\beta^4-8(3\phi+2)(3\pi+8)\beta^2+16(3\phi+2)^2\right]}}_{\gamma_1(\phi)}\\
    &-\underbrace{{2\sin(2k)(4\sin^2(k)+6)((3\pi+8)\beta^2-4(3\phi+2))}}_{\gamma_2(\phi)}\\
    &-\underbrace{{\sin^2(2k)(16\sin^4(k)+48\sin^2(k)+36)}}_{\gamma_3(\phi)}\\
    &=\gamma_1(\phi)-\gamma_2(\phi)-\gamma_3(\phi)
\end{align}}

Applying Equations 1-3 to the denoted sub-functions $\gamma_1(\phi),\gamma_2(\phi),\gamma_3(\phi)$, we get 

\begin{align}
    \gamma_1(\phi)&=(3 \pi+8)^{2} \beta^{4}\left(\cos ^{2}(2 \phi)-1\right)+8(3 \phi+2)\left((3 \pi+8) \beta^{2}-2(3 \phi+2)\right)\\
    \gamma_2(\phi)&=-2(8 \cos (2 \phi)+2\sin(2\phi)\cos(2\phi))\left((3 \pi+8) \beta^{2}-4(3 \phi+2)\right)\\
    &=-4\cos(2\phi)(\sin(2\phi)+2)\left((3\pi+8)\beta^2-4(3\phi+2)\right)\\
    \gamma_3(\phi)&=-\cos ^{2}(2 \phi)\left(16 \sin \left(\phi+\frac{\pi}{4}\right)^{4}+48 \sin ^{2}\left(\phi+\frac{\pi}{4}\right)+36\right)\\
    &=-\cos^2(2\phi)\left(4\sin^2(2\phi)+6\right)^2
\end{align}

Putting this together, a simplified expression is

\begin{align}
    1-w_p(\phi)^2&=\frac{\gamma_1}{(3\pi+8)^2\beta^4\cos^2(2\phi)}+\frac{\gamma_2}{(3\pi+8)^2\beta^4\cos^2(2\phi)}+\frac{\gamma_3}{(3\pi+8)^2\beta^4\cos^2(2\phi)}\\
    &=\underbrace{\frac{\cos^2(2\phi)-1}{\cos^2(2\phi)}}_{\frac{-(1-\cos^2(2\phi))}{\cos^2(2\phi)}=\frac{-\sin^2(2\phi)}{\cos^2(2\phi)}=\boxed{-\tan^2(2\phi)}}+\frac{8(3\phi+2)((3\pi+8)\beta^2-2(3\phi+2))}{(3\pi+8)^2\beta^4\cos^2(2\phi)}\\
    &-\frac{4(\sin(2\phi)+2)((3\pi+8)\beta^2-4(3\phi+2))}{(3\pi+8)^2\beta^4\cos(2\phi)}-\frac{\left(4\sin^2(2\phi)+6\right)^2}{(3\pi+8)^2\beta^4}
\end{align}

Because $y_p(\phi)=\sqrt{1-w_p(\phi)^2}\sqrt{\frac{\mu}{r_0}}\cot(\phi+\pi/4)$, by substitution

{\color{red}\tiny\begin{equation}
    y_p(\phi)=\sqrt{-\tan^2(2\phi)+\frac{8(3\phi+2)((3\pi+8)\beta^2-2(3\phi+2))}{(3\pi+8)^2\beta^4\cos^2(2\phi)}-\frac{4(\sin(2\phi)+2)((3\pi+8)\beta^2-4(3\phi+2))}{(3\pi+8)^2\beta^4\cos(2\phi)}-\frac{\left(4\sin^2(2\phi)+6\right)^2}{(3\pi+8)^2\beta^4}}\sqrt{\frac{\mu}{r_0}}\cot(\phi+\pi/4)
\end{equation}}