\section{Bisection and Newton-Raphson method for finding roots}

\subsection{Bisection method}

We will use the bisection method to find the roots for $w_p\,'(\phi)$ 
in order to minimize $w_p(\phi)$ and find when $y_p(\phi)=0$ for a given tolerance $\epsilon$. 
The requirement for the bisection method is that the function in question is continuous on the closed interval $[a,b]$\footnote{\href{https://mathworld.wolfram.com/Bisection.html}{Bisection method - Wolfram MathWorld}}
, so the roots of both $w_p\,'(\phi)$ and $y_p(\phi)$ can be found.
In Eq. 40 from the presentation, $w_p\,'(\phi)$ was not expressed in terms of $\beta$, 
so the following expression will be used:

\begin{equation}
    w_p\,'(\phi)=\frac{2(\sin^2(2\phi+\pi/2)-3)\sin(2\phi+\pi/2)}{(3\pi+8)\beta^2\sin^3(2\phi+\pi/2)}
    -\frac{\cos(2\phi+\pi/2)\left(1-\frac{4}{(3\pi+8)\beta^2}(3\phi+2)\right)}{r_0\sin^3(2\phi+\pi/2)}
\end{equation}

\subsubsection{Minimizing \texorpdfstring{$w_p(\phi)$}{}}

\subsubsection{Roots for \texorpdfstring{$y_p(\phi)$}{}}

\subsection{Newton-Raphson method}

This method cannot be used to solve for the root of $y_p(\phi)$ because although the function exists in the domain $\phi\in[0,\pi/4)$,
$\lim_{\phi\to 0^+}y_p\,'(\phi)$ and $\lim_{\phi\to \pi/4^-}y_p\,'(\phi)$ do not exist, so the function is not differentiable on the entire interval\footnote{\href{http://amsi.org.au/ESA_Senior_Years/SeniorTopic3/3j/3j_2content_2.html}{Newton's method}}.

From observing the graph of $y_p(\phi)$ on Page 6, this is confirmed visually. Since the endpoints of $y_p(\phi)$ on this interval are its two roots,
they cannot be found through this method. The bisection method works, however since the only requirement is continuity of $y_p(\phi)$ on the interval.

\subsubsection{Minimizing \texorpdfstring{$w_p(\phi)$}{}}

The code used to implement both algorithms is located here.

\section{Curve fitting}