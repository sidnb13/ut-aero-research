\section{Bisection and Newton-Raphson method for finding roots}

\subsection{Bisection method}

Let $f(x)\in \mathbb{R}$ be continuous on the interval where $x,c\in [a,b]$. Also define $\epsilon_0=|a-b|$ as the initial error. The approach to the bisection method involves using a finite number of iterations where the original domain interval $[a,b]$ (and likewise the error) is continuously halved until the desired tolerance $\epsilon$ and thusly the root of the function in question is found.
From, this we can conclude that the required precision/tolerance is $\epsilon=|c_n-c|$ where $f(c)=0$ and $c_n$ is the upper or lower bound of the interval at the $n$th iteration.
Finding an upper bound for the value of $n$ eliminates the need for an infinite loop and prevents divergence of the bisection algorithm. The upper bound for $n$ is expressed as \footnote{\href{https://en.wikipedia.org/wiki/Bisection_method}{Wikipedia - Bisection method}}

\begin{equation}
    |c_n-c|\leq \frac{|b-a|}{2^n}\implies n\leq \log(\frac{\epsilon_0}{\epsilon}).
\end{equation}

We will use the bisection method to find the roots for $w_p\,'(\phi)$ in order to minimize $w_p(\phi)$ and find when $y_p(\phi)=0$ for a given tolerance $\epsilon$.
In Eq. 40 from the presentation, $w_p\,'(\phi)$ was not expressed in terms of $\beta$, so the following expression will be used in the algorithms:


\begin{equation}
    w_p\,'(\phi)=\frac{2(\sin^2(2\phi+\pi/2)-3)\sin(2\phi+\pi/2)}{(3\pi+8)\beta^2\sin^3(2\phi+\pi/2)}
    -\frac{\cos(2\phi+\pi/2)\left(1-\frac{4}{(3\pi+8)\beta^2}(3\phi+2)\right)}{r_0\sin^3(2\phi+\pi/2)}
\end{equation}

\subsubsection{Minimizing \texorpdfstring{$w_p(\phi)$}{}}

\subsubsection{Roots for \texorpdfstring{$y_p(\phi)$}{}}

\subsection{Newton-Raphson method}

The implementation of the Newton-Raphson algorithm used here was derived from the pseudocode on the
corresponding Wikipedia article\footnote{\href{https://en.wikipedia.org/wiki/Newtons_method}{Wikipedia - Newton's method}}.
This method cannot be used to solve for the root of $y_p(\phi)$ because although the function exists in the domain $\phi\in[0,\pi/4)$,
$\lim_{\phi\to 0^+}y_p\,'(\phi)$ and $\lim_{\phi\to \pi/4^-}y_p\,'(\phi)$ do not exist, so the function is not differentiable on the entire interval.

From observing the graph of $y_p(\phi)$ on Page 6, this is confirmed visually. Since the endpoints of $y_p(\phi)$ on this interval are its two roots,
they cannot be found through this method. The bisection method works, however since the only requirement is continuity of $y_p(\phi)$ on the interval.

\subsubsection{Minimizing \texorpdfstring{$w_p(\phi)$}{}}

The code used to implement both algorithms is located here.

\section{Curve fitting}